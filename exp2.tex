%%%%%%%%%%%%%%%%%%%%%%%%%%%%%%%%%%%%%%%%%
% Journal Article
% LaTeX Template
% Version 1.4 (15/5/16)
%
% This template has been downloaded from:
% http://www.LaTeXTemplates.com
%
% Original author:
% Frits Wenneker (http://www.howtotex.com) with extensive modifications by
% Vel (vel@LaTeXTemplates.com)
%
% License:
% CC BY-NC-SA 3.0 (http://creativecommons.org/licenses/by-nc-sa/3.0/)
%
%%%%%%%%%%%%%%%%%%%%%%%%%%%%%%%%%%%%%%%%%

%----------------------------------------------------------------------------------------
%	PACKAGES AND OTHER DOCUMENT CONFIGURATIONS
%----------------------------------------------------------------------------------------

\documentclass[14pt,twocolumn ]{article}

\usepackage{blindtext} % Package to generate dummy text throughout this template
\usepackage{adjustbox}
\usepackage[sc]{mathpazo} % Use the Palatino font
\usepackage[utf8]{inputenc} % Unicode
\usepackage[T1]{fontenc} % Use 8-bit encoding that has 256 glyphs
\linespread{1.05} % Line spacing - Palatino needs more space between lines
\usepackage{microtype} % Slightly tweak font spacing for aesthetics
\usepackage{booktabs}
\usepackage{enumitem}
\usepackage[separate-uncertainty=true]{siunitx}

\usepackage{multicol}
\usepackage{caption}
\usepackage{bm}
\newenvironment{notalistitem}
{\begin{itemize}
\renewcommand{\labelitemi}{$\triangleright$} 
\item}

\usepackage{geometry}% http://ctan.org/pkg/geometry
\usepackage{graphicx}


\usepackage[english]{babel} % Language hyphenation and typographical rules

%\usepackage[hmarginratio=1:1,top=32mm,columnsep=15pt,a4paper,margin=1in, footskip=0.25in]{geometry} % Document margins
%\usepackage[hang, small,labelfont=bf,up,textfont=it,up]{caption} % Custom captions under/above floats in tables or figures
%\usepackage{booktabs} % Horizontal rules in tables

\usepackage{lettrine} % The lettrine is the first enlarged letter at the beginning of the text

\usepackage{enumitem} % Customized lists
\setlist[itemize]{noitemsep} % Make itemize lists more compact

\usepackage{abstract} % Allows abstract customization
\renewcommand{\abstractnamefont}{\normalfont\bfseries} % Set the "Abstract" text to bold
\renewcommand{\abstracttextfont}{\normalfont\small\itshape} % Set the abstract itself to small italic text

\usepackage{titlesec} % Allows customization of titles
\renewcommand\thesection{\Roman{section}} % Roman numerals for the sections
\renewcommand\thesubsection{\roman{subsection}} % roman numerals for subsections
\titleformat{\section}[block]{\large\scshape\centering}{\thesection.}{1em}{} % Change the look of the section titles
\titleformat{\subsection}[block]{\large}{\thesubsection.}{1em}{} % Change the look of the section titles

\usepackage{fancyhdr} % Headers and footers
\pagestyle{fancy} % All pages have headers and footers
\fancyhead{} % Blank out the default header
\fancyfoot{} % Blank out the default footer
\fancyhead[C]{Experimento 2 - PS6 } % Custom header text
\fancyfoot[C]{\thepage} % Custom footer text

\usepackage{titling} % Customizing the title section

\usepackage{hyperref} % For hyperlinks in the PDF
\pdfsuppresswarningpagegroup=1
%----------------------------------------------------------------------------------------
%	TITLE SECTION
%----------------------------------------------------------------------------------------

\setlength{\droptitle}{-6\baselineskip} % Move the title up
%\begin{center}
\pretitle{\centering \Huge\bfseries }  % Article title formatting
\posttitle{\begin{center} \end{center} }% Article title closing formatting

\title{Resistividade elétrica} % Article title

\author{
\textsc{Raylsson Magalhães \,\, Gustavo judice} \\[1ex] % Your name
\normalsize Universidade Federal de Minas Gerais\\ % Your institution
%\normalsize \href{mailto:john@smith.com}{john@smith.com} % Your email address
%\and % Uncomment if 2 authors are required, duplicate these 4 lines if more
%\textsc{Jane Smith}\thanks{Corresponding author} \\[1ex] % Second author's name
%\normalsize University of Utah \\ % Second author's institution
%\normalsize \href{mailto:jane@smith.com}{jane@smith.com} % Second author's email address
}
\date{PS6 \\ \today} % Leave empty to omit a date
\renewcommand{\maketitlehookd}{%
\noindent\rule[0.5ex]{\linewidth}{1pt}
\begin{abstract}
\noindent Cada material condutor possui uma certa resistividade associada a ele e se deve pelo comportamento do sólido em escalas microscópicas. Esse experimento tem como objetivo analisar a resistividade de um fio ao relacionar sua resistência  com diversos de seus comprimentos e posteriormente a partir da resistividade encontrada identificar o tipo de condutor e material estudado.   % Dummy abstract text - replace \blindtext with your abstract text
\end{abstract}
\noindent\rule[0.5ex]{\linewidth}{1pt}
}
\usepackage{systeme}
%----------------------------------------------------------------------------------------

\begin{document}

% Print the title
\maketitle

%----------------------------------------------------------------------------------------
%	ARTICLE CONTENTS
%----------------------------------------------------------------------------------------
%\begin{multicols}{2}
\section{Introdução}



\lettrine[nindent=0em,lines=3]{Q}uando temos um circuito simples sem nenhum nó (ou seja, sem nenhuma bifurcação) podemos analisar a relação entre a resistência de seus resistores e a voltagem pela corrente que percorre o circuito usando a famosa relação de Ohm:


\begin{equation}
\label{eq:resist}
 V =  R \, I
\end{equation}



%------------------------------------------------
Entretanto, quando temos circuitos mais complexos com diversas malhas e bifurcações temos que usar leis de conservação de energia e carga para encontrar todos os valores desejados.Essas regras são conhecidas como Regras de Kirchhoff, nomeada pelas grandes contribuições de Gustav Kirchhoff para o entendimento dos circuitos elétricos. Essas regras podem ser resumidas da seguinte forma:



\begin{notalistitem}
Conservação de Energia: Em uma malha do circuito, a soma das forças eletromotrizes é igual a soma das diferenças de potencial nos demais elementos constituintes da malha:


\begin{equation}
\label{eq:resist}
     \sum_{k=1}^{N} U_{k}=0 
\end{equation}

$U_k$ nesse caso é a diferença de pontencial em uma parte do percurso fechado.

\end{notalistitem}

\begin{notalistitem}
Conservação de carga:  A soma das correntes que chegam de um nó de um circuito é igual a soma das correntes que saem desse nó:

\end{notalistitem}



\begin{equation}
\label{eq:resist}
     \sum_{k=1}^{N} i_{k}= 0 
\end{equation}

$i_k$ nesse caso é uma das correntes com indice variando que passa entra e sai pelo nó ao longo do percurso fechado.  


Assim, o objetivo do experimento é validar se as regras de Kirchhoff são realmente válidas. Para isso é iremos montar um circuito semelhante a figura 1.

Ao aplicar as regras de Kirchhoff temas três equações e três variaveis que podem ser resolvidas facilemnte e a partir da resolução das equações relacionar com os valores encontrados experimentalmente.





    
\begin{figure}
     
      \includegraphics[width=8cm,height=5cm\textbf{}]{01.png}
    \caption{Circuito composto por 3 malhas - ABEFA, BCDEB e ABCDEFA e dois nós B e E}
    \label{fig:my_label}
\end{figure}





\section{Metodologia}


Foram utilizados os seguintes materiais para realizar o experimento:

\begin{itemize}
\item Fonte de tensão com valores $\varepsilon_{1} = \SI{6.5}{\volt}$ e $\varepsilon_{2} = \SI{3}{\volt}$ respectivamente.
\item Multímetro e painel para conexões
\item Resistores para montar o circuito tendo suas respectivas resistências : $R_1 = \SI{6.5}{\ohm}$, $R_2= \SI{6}{\ohm} $ e $R_3 = \SI{9}{\ohm}$ respectivamente. 
	\end{itemize}
% Dummy text



%Text requiring further explanation\footnote{Example footnote}.

%------------------------------------------------

\section{Resultados}

Temos três equações no total: 



\begin{equation}
\label{eq:resist}
     I_{1}=I_{2}+I_{3} 
\end{equation}

Ao aplicar a regra dos nóis no ponto B.


\begin{equation}
\label{eq:resist}
\varepsilon_{1}=I_{1} R_{1}+I_{2} R_{2}
\end{equation}
Ao aplicar a regra das malhas para ABEFA


\begin{equation}
\label{eq:resist}
\varepsilon_{2}=-I_{2} R_{2}+I_{3} R_{3}
\end{equation}
Ao aplicar a regra das malhas para BCDEB. 



Ao substituir 4 em 5 ficamos com um sistema linear de 2 incognitas:
\[
\syslineskipcoeff{1}
\systeme*{\varepsilon_{1}= I_{2}R_1 + I_{2} R_2 + I_{3}R_1, \varepsilon_{2}= -I_{2}R_2 + I_{3}R_3}
\]

Ao simplificar o sistema de equações lineares foi possivel encontrar os valores para as correntes e posteriormente os valores de diferença de potencial ao usar a relação de Ohm da 1 equação:

\begin{table}[ht]
\centering
\caption{Valores encontrado por meio das equações ao resolver o sistema linear}
\begin{tabular}[t]{lccr}
\toprule
Corrente $I (\SI{}{\ampere})$    & Voltagem$V(\SI{}{\volt})$ \\
\midrule
I_1 = \SI{0.7623}{\ampere} & V_1 = 4.955 \\
I_2 = \SI{0.2574}{\ampere} & V_2 = 1.544 \\
I_3 = \SI{0.5049}{\ampere} & V_3 = 4.545 \\
    
\bottomrule
\end{tabular}
\end{table}%

Ao montar o circuito no laboratório online conforme a imagem da figura 1 temos os resultados das correntes verificadas experimentalmente.

\begin{figure}
     
      \includegraphics[width=8cm,height=5cm\textbf{}]{02.png}
    \caption{Circuito montado em laboratório testado e compravado internacionalmente.}
    \label{fig:my_label}
\end{figure}



%%\begin{table}
%\caption{Example table}
%\centering
%\begin{tabular}{llr}
%\toprule
%\multicolumn{2}{c}{Name} \\
%\cmidrule(r){1-2}
%First name & Last Name & Grade \\
%\midrule
%J%ohn & Doe & $7.5$ \\
%R%ichard & Miles & $2$ \\
%\bottomrule
%%\end{tabular}
%\end{table}



%\begin{equation}
%\label{eq:emc}
%e = mc^2
%\end{equation}

%\blindtext % Dummy text

%\end{multicols}

%\subsection{Gráfico}

%\begin{figure}[h]
%  \includegraphics[width=\textwidth,height=8cm]{rVl2.png}
%  \caption{This is a tiger.}
%\end{figure}


%Essse grafico deveria 
%\begin{figure*}
	  %  \includegraphics[width=\textwidth,height=8cm]{rVl2.png}
	   %  \caption{Gráfico com os dados plotados e a regresão linear feita}

%\end{figure*}


%Essse grafico MINGAU deveria estar acima de mim.
%\begin{multicols}{2}


%------------------------------------------------

\section{Discussão}

\subsection{Os valores encontrados}

Vendo a tabela de valores de resistividade de alguns materiais, dada no guia de experimento,
\begin{table}[ht]
\centering
\caption{Exemplos de valores da resistividade de alguns materiais}
\begin{tabular}[t]{lccr}
\toprule
Material    &Resistividade $ \rho  (\SI{}10^{-8}\,\SI{}{\ohm \, \meter}) $ \\
\midrule
   Cobre&$\SI{1,72(1)}{}$\\
   Ouro&$\SI{2,44(2)}{}$\\
   Alumínio&$\SI{2,82(2)}{}$\\
   Tungstênio&$\SI{5,6(1)}{}$\\
   Ferro&$\SI{10,0(3)}{}$\\
    Liga cobre-níquel&$\SI{44(1)}{}$\\
    Liga níquel-cromo&$\SI{100(5)}{}$\\
    Liga Kanthal&$\SI{139(4)}{}$\\
    Carbono&$\cong3500$\\

    
\bottomrule
\end{tabular}
\end{table}%

encontramos a liga Kanthal com $$ \rho =  \SI{1,39(4)}10^{-6}\,\SI{}{\ohm \, \meter} $$ Este valor está razoavelmente próximo do que encontramos, especialmente se considerarmos a larga incerteza da nossa medida. O que nos leva a concluir que o material sob investigação provavelmente é feito de liga Kanthal.


\subsection{A estimativa do erro}

Um dos aspectos que mais chamou nossa atenção foi a propagação de erro. Notamos que nos foi dado o diâmetro do fio:
$$ {\phi} =  \SI{0,25(5)}\,\SI{}{\mm} $$
Ora, isto corresponde a 20\% de margem de erro. Fazemos então o cálculo da área, com a correspondente incerteza.

$$\Delta A = \frac{dA}{d\phi}\Delta \phi=\frac{\pi \phi}{2}\Delta \phi= (1.96)10^{-8}  \SI{}{\, \meter\squared}$$ 

Obtemos com isto 39.9\% de erro, uma expressiva propagação de erro.
Por outro lado, tão alto percentual de erro não chega a seriamente ameaçar a estimativa de material, pois ao se tratar de resistividade os materiais podem variar em muitas ordens de grandeza. Mesmo com esta alta taxa, na tabela de referências só a liga níquel-cromo, com $\SI{1.00(5)}10^{-6}\SI{}{\ohm \, \meter}$, é uma possível alternativa.

%Concluímos: ALL HAIL BAETA

\section{Conclusão}

O experimento foi um sucesso e mostrou muito bem a relação entre a resistividade de um material (em temperatura ambiente) com a sua resistência e comprimento.  
% \begin{equation}
% \label{eq:delA}
% \Delta A = \frac{dA}{d\phi}\Delta \phi=\frac{\pi d}{2}\Delta \phi= (1.96)10^{-8}
% \end{equation}
% \section{Conclusão}

%----------------------------------------------------------------------------------------
%	REFERENCE LIST
%----------------------------------------------------------------------------------------

%\begin{thebibliography}{99} % Bibliography - this is intentionally simple in this template

%\bibitem  [Figueredo and Wolf, 2009]{Figueredo:2009dg}
%        Figueredo, A.~J. and Wolf, P. S.~A. (2009).
%\newblock All hail Baeta - a cross-cultural
%  study.
%\newblock{\em Human Nature}, 20:317--330.

%\end{thebibliography}
\section{Gráfico}
%\begin{center}

%    \includegraphics[width=\textwidth,height=8cm]{rVl2.png}
%\captionof{figure}{Gráfico com os dados plotados e a regresão linear feita}\label{fg1}      only if needed  
%    \caption{Gráfico com os dados plotados e a regresão linear feita}
%\end{center}

\begin{center}
    

\begin{minipage}{\textwidth}% to keep image and caption on one page
\makebox[\textwidth]{%        to center the image
  \includegraphics[width=\textwidth,height=8cm\textbf{}]{rVl2.png}}
\captionof{figure}{Gráfico com os dados plotados e a regresão linear feita.}\label{visina8}%      only if needed  
\end{minipage}

\end{center}
%\begin{figure*}[h]
% maybe other stuff

%\includegraphics[width=\textwidth,height=8cm]{rVl2.png}% example only, could also be \adjustimage
% \caption{Gráfico com os dados plotados e a regresão linear feita}
% maybe other stuff
%\end{figure*}
%\begin{figure}
        %\centering
%	     \includegraphics[width=\textwidth,height=8cm]{rVl2.png}
%	     \caption{Gráfico com os dados plotados e a regresão linear feita}

%\end{figure}
%----------------------------------------------------------------------------------------
%\end{multicols}

\end{document}


